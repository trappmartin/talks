
\begin{frame}{Bayesian Parameter Learning}
  \begin{itemize}
    \item The key insight for Bayesian parameter learning [Zhao2016, Vergari2019] is that \emph{sum nodes can be interpreted as latent variables} $\fcolorbox{CBYellow}{background}{Z}_\SumNode$, clustering data instances.
    \pause
    \item Consider $\z$ to be a vector containing a state for each $\SumNode$ in the SPN. Then a $\SPT$ is a so-called induced tree [Zhao2016] which is induced by $\z$.
  \end{itemize}
  \pause
  \begin{figure}
  \centering{
    \includestandalone[width=0.7\textwidth]{InducedTree}
    }
\end{figure}
\end{frame}
%
%\begin{frame}{Bayesian Parameter Learning}
%  \begin{itemize}
%    \item It is now conceptually straightforward to extend an SPN to a Bayesian setting.
%  \end{itemize}
%  \pause
%  \begin{figure}
%
%\centering
%
%\begin{tikzpicture}
%
%  \node[obs, fill=lightbackground, draw=white] (x) {$\xnd$};
%
%  \node[foreground, draw=CBYellow, latent, fill=background, left=of x] (z) {$z_{\SumNode,n}$};
%
%  \node[foreground, draw=CBPurple, latent, fill=background, right=of x] (t) {$\theta_{\Leaf,d}$};
%
%  \node[foreground, draw=CBGreen, latent, fill=background, below=of z] (w) {$w_{\SumNode}$};
%
%
%
%  \edge {z,t} {x} ; %
%
%  \edge {w} {z} ; %
%
%
%
%  \plate {zs} {(z)(w)} {$\forall \SumNode \in \SumNodes$} ;
%
%  \plate {t} {(t)} {$\forall \Leaf \in \Leaves$} ;
%
%
%
%  \plate [inner xsep=0.3cm] {xyt} {(x)(t)} {$d\!\in\!1\!:\!D$} ;
%
%  \plate [inner xsep=0.3cm] {xz} {(x)(z)(xyt.north west)} {$n\!\in\!1\!:\!N$} ;
%
%\end{tikzpicture}
%
%\caption{Generative model for Bayesian parameter learning.} \label{fig:BSPN}
%
%\end{figure}
%
%\end{frame}
